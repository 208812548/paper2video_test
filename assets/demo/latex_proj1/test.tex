\documentclass[10pt]{article}
\usepackage[verbose, a4paper, hmargin=2.5cm, vmargin=2.5cm]{geometry}

\usepackage{fontspec}
\usepackage{ctex}
\usepackage{paratype}


\usepackage{amsmath}
\usepackage{amsfonts}
\usepackage{amssymb}
\usepackage{esint}

\usepackage{graphicx}
\usepackage[export]{adjustbox}
\usepackage{mdframed}
\usepackage{booktabs,array,multirow}
\usepackage{adjustbox}
\usepackage{tabularx}
\usepackage{hyperref}
\hypersetup{colorlinks=true, linkcolor=blue, filecolor=magenta, urlcolor=cyan,}
\urlstyle{same}
\usepackage[most]{tcolorbox}
\definecolor{mygray}{RGB}{240,240,240}
\tcbset{
  colback=mygray,
  boxrule=0pt,
}
\graphicspath{ {./images/} }
\newcommand{\HRule}{\begin{center}\rule{0.9\linewidth}{0.2mm}\end{center}}
\newcommand{\customfootnote}[1]{
  \let\thefootnote\relax\footnotetext{#1}
}

\begin{document}
\section*{基于学生数据的缺考分析及预测}

\section*{摘要}

随着对学生管理工作的精细化和精准化, 及时掌握学生的学习动态, 特别是预测学生是否可能缺考这一行为对于提高课程的通过率和教学质量至关重要。为了帮助教师提前预知高风险缺考学生, 提升教学效果, 本研究针对学习网课程的学生数据, 构建了基于机器学习的缺考预测模型。研究首先采用皮尔逊等相关系数方法分析不同指标特征与学生缺考的关系。随后, 结合数据特征构建适合预测任务的数据处理流程, 为模型训练提供高质量输入。

进一步, 本研究使用逻辑回归、支持向量机、梯度上升机和随机森林四种机器学习模型对学生是否缺考进行预测,并通过准确率、精确率、召回率和 F1 分数等指标进行评估。结果表明, 在单课程上使用随机森林总体上表现最优, 准确率达到了 83\%,在缺考学生的识别方面具有较高的精确率 (70\%左右)。然而,由于数据质量以及类别不平衡等问题, 模型在缺考学生的召回率预测上表现稍差 (48\%左右)。

为精准识别缺考学生, 我们进一步聚焦于前置学历限制下的全课程学习行为数据, 并通过随机森林模型进行预测分析。通过限制前置学历后, 模型对缺考学生的识别能力显著增强。精确率提升 15\%左右,召回率提升 20\%左右,这反映模型在精确率和召回率之间达到更优平衡,综合性能显著增强。

本研究的意义在于通过数据驱动的方式, 为开放大学提供一种有效的工具, 以辅助教师在教学过程中提前发现高风险缺考学生, 进而对学生开展指导与访谈工作。

\section*{目录}

第一章 引言. 3

1.1 研究背景. 3

1.2 研究问题. 3

1.3 研究目的. 2

1.4 研究意义 2

第二章 文献综述. 4

第三章 研究方法. 6

3.1 指标设计. 6

3.2 数据抓取. 7

3.3 相关性分析. 8.8

3.3.1 相关性系数介绍 8.

3.3.2 相关性结果分析 10

3.3.3 总结 13

3.4 不同指标与实考率的关系 错误!未定义书签。

3.5 特征工程构建/数据标准化处理 错误!未定义书签。

3.6 建模预测. 错误!未定义书签。

第四章 研究结果及结论. 错误!未定义书签。

4.1 结果展示及分析. 错误!未定义书签。

4.2 前置学历为本科的预测结果 错误!未定义书签。

4.3 结论与局限 错误!未定义书签。

4.3.1 结论 错误!未定义书签。

4.3.2 局限 错误!未定义书签。

第五章 参考文献. 错误!未定义书签。

\section*{第一章 引言}

\section*{1.1 研究背景}

成人教育在提升社会整体素质、促进职业发展和个人成长方面发挥着重要作用。然而, 由于成人学生多为在职人员, 他们面临工作、家庭与学习之间的多重压力, 缺考现象较为普遍。这不仅影响到成人教育的教学效果与课程通过率, 也可能导致学生学习积极性下降, 甚至中途辍学。因此, 如何有效预测成人学生的缺考情况, 并在关键时刻提供干预措施, 成为成人教育机构亟待解决的问题。

传统的学生管理方法主要依靠线下督促与人工跟踪, 但这种方式在面对庞大而分散的成人学生群体时效率低下、成本较高。而借助大数据与机器学习技术, 分析学生的学习行为与成绩表现, 可以更准确地识别出缺考风险较高的学生, 从而为老师提供科学、自动化的决策支持。

本研究以学生数据为基础, 选取了性别、年纪、课程编号、课程讨论次数、 形考分数等特征, 采用随机森林、逻辑回归、轻量梯度上升机和支持向量机等机器学习模型进行预测,并通过准确率、精确率、召回率与 \(\mathrm{F}1\) 分数等指标评估模型效果。本研究旨在为成人教育机构提供学生缺考预测支持, 帮助减少缺考现象, 提高课程完成率与教育质量,同时探索更适合成人学生特性的预测模型与方法。

\section*{1.2 研究问题}

在成人教育领域, 由于学生身份的特殊性 (多为在职人员或社会人士), 学习过程中存在着明显的非全日制特点, 这使得缺考现象频繁发生。缺考不仅会影响学生个人的学习进度和教育机构的教学效果, 还可能导致学生最终放弃学业。 因此, 本研究试图解决以下关键问题:

1. 哪些因素能影响成人学生缺考?

分析学生的性别、年纪、课程讨论次数、期中考试分数等特征与缺考之间的关联, 找出对缺考影响最显著的特征。

2. 能否通过机器学习模型有效预测成人学生的缺考?

针对成人学生的数据特点, 探索适用于缺考预测的机器学习模型, 并评估其效果, 识别高缺考风险的学生。

3. 现有模型在缺考学生的分类上存在哪些问题? 如何改进?

针对实验结果中模型在缺考学生识别上的不足 (如召回率较低), 提出模型优化方向或改进策略,以提高缺考学生的识别能力。

\section*{1.3 研究目的}

本研究旨在针对成人教育学生缺考现象,通过机器学习模型对学生缺考情况进行预测与分析。由于成人学生面临工作与学习的双重压力, 缺考现象显著影响学习效果与课程完成率, 因此本研究的主要目的是构建有效的缺考预测模型, 并评估其在成人教育数据上的表现,为后续相关研究提供基础和参考。

具体而言, 本研究的目的包括:

1. 识别影响缺考的重要因素

通过分析学生的性别、年纪、课程讨论次数、期中考试分数等特征,找出对缺考最有影响的关键变量,帮助理解成人学生缺考的潜在原因。

2. 比较不同机器学习模型的预测效果

采用决策树、随机森林、逻辑回归、K 均值聚类和支持向量机等多种模型, 对学生缺考进行预测,并通过准确率、精确率、召回率和 F1 分数等指标评估模型性能, 筛选出表现最优的模型。

3. 分析改进模型在缺考学生识别上的不足

针对模型在缺考学生分类上的不足, 例如召回率较低的问题, 提出可能的改进方向, 以提高未来模型对缺考学生的预测能力。

通过本研究, 希望能够为成人教育领域提供一套有效的数据分析与预测方法, 帮助老师提前预知高风险缺考人群, 并且为后续研究与实践提供数据支持与技术参考。

\section*{1.4 研究意义}

随着成人教育规模的不断扩大, 如何有效提升成人学生的课程完成率与学习效果, 已成为教育机构与教师关注的重点。由于成人学生普遍面临工作、家庭与学习的多重压力,缺考现象频繁发生,影响了成人教育的教学效果和整体质量。 因此, 本研究具有以下重要意义:

1. 理论意义

本研究通过多种机器学习模型对成人学生缺考情况进行预测与分析, 并探索模型在不同特征组合下的预测性能, 丰富了成人教育领域学生行为预测的研究内容。同时,本研究进一步揭示了成人学生缺考现象的影响因素,为深入理解缺考问题提供了数据支持与理论依据。

2. 实践意义

本研究提出了一套基于数据驱动的缺考预测方法, 通过识别高风险缺考人群, 帮助教师在考试前对可能缺考的学生进行重点关注, 提高教学管理效率。此外, 研究结果也为未来开发更加智能化的成人教育管理系统提供了参考, 有助于提升成人教育的服务水平和课程完成率。

\section*{3. 方法意义}

本研究采用决策树、随机森林、逻辑回归、K 均值聚类和支持向量机等多种机器学习模型, 比较了各模型在成人学生缺考预测中的表现, 提供了不同模型在这一领域的适用性分析, 为后续类似研究中模型的选择与应用提供了方法论指导。

\section*{第二章 文献综述}

成人教育因其受众具有年龄跨度大、学习动机多样、学习方式灵活等特点, 使得学生的行为与传统全日制教育中的学生行为存在较大差异。学生缺考问题在成人教育领域普遍存在,这不仅影响学习效果和学生的最终成绩,还对学校的教学管理产生不利影响。因此, 如何有效预测学生缺考逐渐成为近年来教育实践领域的研究关注点。本章将从成人教育与学生行为分析、缺考相关行为预测的研究方法以及相关机器学习技术的应用几个方面进行综述。

\section*{一、成人教育与学生行为分析}

成人教育因其受众具有年龄跨度大、学习动机多样、学习方式灵活等特点, 使得学生的行为与传统全日制教育中的学生行为存在较大差异。文献指出, 影响成人学生学习参与度和出勤率的因素主要包括工作压力、学习动机、在线学习环境及课程难度等内外部因素错误!未找到引用源。

\section*{二、缺考相关行为预测的研究方法}

现有缺考预测研究主要集中在两方面: 一是基于统计方法的传统研究, 二是基于机器学习的现代方法。传统方法主要通过回归分析和方差分析探讨影响缺考的主要因素, 如学生的个人特征、课程难度和教师教学风格等。然而, 传统方法难以处理高维非线性特征之间的复杂关系, 预测准确率有限。

近年来, 随着数据驱动技术的发展, 基于机器学习的预测方法逐渐兴起。这类方法利用学生的历史学习数据, 训练分类模型来判断学生旷课或者缺考的行为。例如, Lindita 等人研究采用逻辑回归和随机森林等传统机器学习模型进行学生旷课预测,并取得了较好的效果需要!未找到引用源。

\section*{三、机器学习在教育预测中的应用}

机器学习技术在教育预测领域得到了广泛应用, 常用于学生成绩预测、辍学风险评估等任务。与传统统计方法相比, 机器学习能够捕捉特征之间的复杂非线性关系, 提升预测准确性。特别是在类别不平衡问题中, 采用集成学习方法 (如随机森林 \({}^{{铺设},l}\) 用源。轻量梯度上升机 \({}^{{铺设},l}\) 未找到引用源。有助于提高少数类(如缺考学生)的预测效果。

为了评估模型的性能, 本研究采用了准确率 (Accuracy)、精确率

(Precision)、召回率(Recall)和 F1 分数 (F1-score) 四个常见指标 \({}^{{带限}!{未找到}}\) 引用源。一错误!未找到引用源。准确率衡量模型整体预测的正确性, 精确率和召回率分别反映了模型对缺考或者未缺考学生的识别准确性和覆盖率, 而 F1 分数作为二者的综合指标, 能更全面评估模型在少数类上的表现。

本章综述了成人教育领域学生行为分析的相关研究, 介绍了缺考预测的传统与现代方法, 并探讨了机器学习技术在教育领域的广泛应用。这些研究为本研究提供了理论依据与方法参考。接下来将详细介绍特征指标、数据预处理、特征工程及模型构建过程。

\section*{第三章 研究方法}

\section*{3.1 指标设计}

1. 是否缺考 (result)

是否缺考是一个需要预测的指标, 用于直接表明学生是否实际出现缺考情况, 使用整数 0 - 缺考, 1-未缺考, 是整个研究的核心关注结果。

2. 年纪 (age)

学生的年纪大小。

3. 专业层次 (specialty\_level)

学生的所在读的学历层次。

4. 在读时长 (grade)

指学生注册并参与学习的时间长度, 通常以年或学期为单位。这个指标可以反映出学生的学习进程阶段、学习动机与坚持程度。

5. 形考成绩 (formal\_score)

指学生在平时的学习过程中的考核成绩, 该指标能够反映学生的学习态度与积极性、知识的掌握程度以及时间管理能力和学习习惯。

6. 学生课程上线天数 (student\_online\_day\_count)

是指学生在学习平台上累计参与课程学习的天数。这一指标在成人教育领域具有重要意义, 它可以反映学生学习积极性、学习习惯与投入以及学习态度。

7. 行为数 (action\_count)

这是一个综合指标, 包含学生在学习过程中执行的各种行为活动, 如浏览课件、观看视频、参与课程讨论等。该指标综合反映了学生在平台上的学习参与度、 活跃度、学习投入和课程关注度。

8. 学生发帖数量 (topic\_count)

该指标指的是学生在课程讨论区、学习平台或者其他互动平台上发布的帖子数量。这个指标可以反映学生的参与度、学习态度以及与课程内容的互动情况。

9. 学生回帖数量 (reply\_count)

指的是学生在课程讨论平台上, 针对其他学生或课程内容发表的回复或评论数量。该指标可以反映学生在课程中的参与度和互动情况, 通常用于衡量学生的学习积极性和课堂互动性。

10. 课程完成率(finish\_rate)

指学生在学习过程中完成课程内容的比例, 通常用已完成的学习任务与总任务的比值表示。这个指标能够反映学生的学习进度、学习投入以及对课程的态度。

除了上述指标外, 还包括音视频观看时长、测验数量、学生选课数量以及测验提交时隔天数。

\section*{3.2 数据抓取}

\section*{1. 数据来源和构成}

为支持成人教育学生缺考行为预测研究, 数据全面整合教务系统内部多源信息,构建覆盖学生属性、课程属性、考试属性、学习行为四大核心维度的结构化数据集。数据来源于教务管理数据库、在线学习平台日志及考试管理系统,共包含 1300 余门课程、800 多万条记录,具体构成如下:

学生属性数据包含专业层次 (本科/专科等)、年龄、性别、前置学历、入学年份等静态信息, 用于刻画学生背景特征。课程属性数据涵盖课程层次 (公共基础课、专业核心课)、考核形式 (闭卷笔试、论文答辩)、形考成绩占比、历史平均通过率等指标, 反映课程对学生的挑战度与吸引力。考试属性数据聚焦考试时间安排(工作日/节假日)、考试方式(线上监考/线下集中)、考点地理位置等外部约束因素, 识别客观环境对参与度的影响。学习行为数据通过埋点采集高频动态指标, 包括在线学习天数、视频观看总时长、章节完成率、作业提交延迟天数、 讨论区互动频次、模拟测试参与度等细粒度行为序列, 构建从学习投入度到考试决策的行为链。

\section*{2. 数据抓取}

在本研究中, 共抓取了学习网共八百多万条所有课程的学生数据。学习网的数据抓取采用了两种主要的技术手段:CDC(Change Data Capture)和华为云 CDM (Cloud Data Migration),这两种方式分别满足了不同的数据抓取需求。

1. CDC (Change Data Capture) 技术: CDC 是一种高效的数据捕获技术, 通过实时捕获数据库中的变更事件, 并将这些事件转化为流式数据, 从而实现数据的实时同步。FlinkCDC 作为一种流式数据处理框架, 广泛支持多种数据库, 如 MySQL、PostgreSQL 等,并通过读取数据库的事务日志 (如 binlog) 来捕获数据变更事件。

基于目标数据库的 CDC 方案具有以下特点:(1)实时性:相比传统的批量数据提取, CDC 能够在数据产生的瞬间进行同步,从而确保数据的时效性。(2)数据一致性: 通过事务日志捕获变更, CDC 技术能够确保数据的一致性和可靠性, 避免了数据在传输过程中可能出现的丢失或错误。(3)低延迟:由于 CDC 无需全表扫描, 它具有较低的延迟, 能够迅速反映数据库中的变更, 且不会增加数据库的额外负载。(4)业务解耦:CDC 技术无需修改业务模型或数据库结构,便可实现数据同步。这使得系统的开发和运维工作变得更加灵活。

2. 华为云 CDM (Cloud Data Migration): 华为云 CDM 提供同构/异构数据源之间的批量数据迁移服务,支持多种数据源。通过 CDM,数据可以从多个源系统迁移至目标系统,支持跨平台和跨数据库的无缝迁移。

华为云 CDM 的主要特点:(1)高效性:CDM 利用分布式计算框架和并行化处理技术, 能够在保证高效性的同时, 稳定地进行大规模数据迁移任务。(2)增量数据迁移: CDM 能够仅迁移变动数据, 从而减少了传输的带宽和时间, 提升了迁移效率。(3) 事务模式迁移: 在 CDM 作业执行过程中, 若遇到失败, 系统能够自动将数据回滚至作业开始前的状态, 保证数据的一致性与完整性。(4) 多数据源支持: 华为云 CDM 支持多种数据源之间的数据迁移, 灵活应对多种数据环境。

总体而言, CDC 和华为云 CDM 的结合, 充分发挥了两者在实时性与批量处理方面的优势, 为学习网数据的抓取与同步提供了稳定、高效、灵活的技术保障。

\section*{3.3 相关性分析}

\section*{3.3.1 相关性系数介绍}

我们首先通过相关性分析, 对各个特征指标与缺考指标 (result1) 进行单调性分析, 以评估每个特征指标对缺考与否的影响程度。使用的数据为学习网全部课程的学生数据共八百多万条。相关性分析是统计学中用来衡量变量之间线性或非线性关系强度和方向的工具。在本研究中, 使用皮尔逊相关系数 (Pearson)、 斯皮尔曼等级相关系数(Spearman)和肯德尔秩相关系数(Kendall)三种方法, 分析不同特征与缺考之间的相关性, 以揭示可能影响成人学生缺考的关键因素。 以下是对这三种相关性分析方法的详细介绍。

\section*{1. 皮尔逊相关系数}

皮尔逊是一种衡量两个变量之间线性关系强度和方向的统计指标。优点在于简洁且计算效率高,广泛应用于线性关系的分析。局限是对于非线性关系或极端值 (离群点) 敏感,数据需要满足正态分布。它适用于如下的情况:(1) 线性关系, 皮尔逊相关系数仅适用于测量两个变量之间的线性关系。如果两个变量之间存在非线性关系, 皮尔逊相关系数可能无法准确反映其相关性。(2) 连续变量: 皮尔逊相关系数适用于连续型数据 (如测量值、比率等), 不适用于分类数据或秩次数据。(3)正态分布:理想情况下,变量应当服从正态分布,或者至少满足对称分布。(4)无异常值:皮尔逊相关系数对异常值非常敏感,异常值可能会显著影响相关系数的大小和方向。皮尔逊相关系数的图如图 3.1 所示。

2. 斯皮尔曼相关系数

这是一种非参数检验方法, 用于衡量两个变量之间的单调关系, 即变量之间的关系不一定是线性, 但可能是增减趋势的一致性。斯皮尔曼相关系数的计算基于变量的排名, 而不是原始数据值。优点在于不要求数据具有正态分布, 适用于非线性但单调的关系, 适合于小样本或具有异常值的数据。局限在于只能反映单调关系,不能揭示出非单调的复杂关系。它的适用范围:(1)单调关系,斯皮尔曼相关系数适用于测量两个变量之间的单调关系, 无论这种关系是线性还是非线性。只要一个变量增加 (或减少),另一个变量也相应增加 (或减少)。(2) 连续或离散变量:斯皮尔曼相关系数适用于连续型数据和秩次数据 (如等级、排名)。 (3)非正态分布,斯皮尔曼相关系数不要求数据服从正态分布,因此适用于数据分布不明或非正态分布的情况。(4)对异常值不敏感,由于斯皮尔曼相关系数基于秩次而非具体数值, 它对异常值不太敏感。斯皮尔曼相关系数热力图如图 3. 2 所示。

3. 肯德尔相关系数

肯德尔相关系数与斯皮尔曼类似, 也是基于变量的排名来衡量两个变量之间的相关性。它通过计算数据中符合一致性和不一致性的排名对数量来评估相关性。优点在于比斯皮尔曼更稳健, 尤其在样本较小或数据中存在重复值时, 表现更为可靠。局限是计算相对较复杂, 特别是在大规模数据集上, 计算量较大。它的适用范围:(1)数据不满足正态分布,当数据不符合正态分布时,肯德尔相关系数是一个更好的选择。(2)样本量较小,在样本量较小的情况下,肯德尔相关系数能够提供较为稳定的关联性度量。(3)序数变量,当变量是序数变量时, Kendall 相关系数能够有效地评估它们之间的关联性。肯德尔相关系数热力图如图 3.3 所示。

\section*{3.3.2 相关性结果分析}

以下是各个相关性系数的热力图及其分析:

\section*{1. 皮尔森相关系数}

皮尔森相关系数矩阵的热力图如 3.1 所示, 它用于展示不同变量之间的线性相关性。它度量两个变量之间线性关系强度和方向的统计量, 其值介于-1 和 1 之间:1 表示完全正相关,即一个变量增加,另一个变量也增加。-1 表示完全负相关,即一个变量增加,另一个变量减少。0 表示没有线性相关关系。

\begin{center}
\includegraphics[max width=0.9\textwidth]{images/bo_d3vjp13ef24c73d2n190_11_271_1010_1114_718_0.jpg}
\end{center}
\hspace*{3em} 

图 3.1 皮尔逊相关系数热力图

在这张热力图中, 颜色的深浅表示相关系数的大小, 通常颜色越深, 表示相关性越强。颜色的正负表示相关性的方向, 通常蓝色表示正相关, 黄色或接近-1 的颜色表示负相关。从图中可以观察到以下几点: select\_course\_count(选课数量):相关系数为-0.24,这意味着当选课数量增加时,缺考的可能会增加。这可能表明选课数量较多的学生可能更容易缺考, 这可能与学生承担的课程负担有关, 当课程负担过重时, 学生可能会选择性地缺考某些课程。finish\_rate (完成率):相关系数为 0.32 ,这意味着完成率越高,学生缺考的可能性越小。这表明学生如果能够完成更多的课程要求, 他们可能更有可能参加考试, 而不是缺考。 student\_online\_day\_count (学生在线天数) : 相关系数为 0.17 , 这意味着在线天数较多的学生可能更少缺考, 这可能表明他们更投入于学习, 因此更有可能参加考试。action\_count(操作次数):相关系数为 0.08 ,这意味着操作次数较多的学生可能更少缺考,这可能与他们更频繁地参与学习活动有关。 exam\_finish\_count (测验完成数量) : 相关系数为 0.11 , 这意味着测验完成数量较多的学生可能更少缺考,这表明他们更有可能参加并完成考试。

其余的特征与缺考的相关性较低, 表明它们可能不会直接影响学生的缺考行为。然而, 这并不意味着这些特征在所有情况下都不重要, 它们可能与其他因素结合时对缺考有间接影响, 或者在特定的分析模型中发挥作用。在构建预测模型时, 可以考虑包含那些与缺考有中度或轻度相关性的特征, 同时探索其他可能影响缺考的非线性因素或定性因素。

2. 斯皮尔曼相关系数热力图

\begin{center}
\includegraphics[max width=1.0\textwidth]{images/bo_d3vjp13ef24c73d2n190_12_240_1186_1171_762_0.jpg}
\end{center}
\hspace*{3em} 

图 3.2 斯皮尔曼等级相关系数热力图

斯皮尔曼相关系数是衡量两个变量之间单调关系强度的统计量, 其值同样介于 -1 和 1 之间。 1 表示完全正相关, -1 表示完全负相关, 0 表示没有相关性。与皮尔森相关系数不同, 斯皮尔曼相关系数是基于变量值的等级 (顺序) 而不是实际值, 因此它适用于非正态分布的数据或者当关系是单调但不必然是线性的时候。热力图 3.2 的分析如下:select\_course\_count(选课数量):相关性-0.22, 这表明选课数量越多的学生,缺考的可能性也相对较高。finish\_rate (完成率): 相关性系数 0.26 , 这表明完成率较高的学生,缺考的可能性相对较低。 student\_online\_day\_count (在线学习天数): 相关性系数 0.27, 这表明学生在线的时间越长,越对这门课积极感兴趣,缺考的可能性就相对较低。action\_count (活动次数): 相关性系数 0.22, 这表明该学生对课程的操作次数越高, 缺考的可能性就越小。exam\_finish\_count (测验完成数量): 相关性为 0.19, 这表明该学生在课程完成的测验越多,缺考的可能性就越小。video\_duration (视频观看时长):相关性为 0.12 ,表明视频观看时长越长,学生越可能参加考试。

总的来说, 大多数特征与缺考之间的相关性较弱, 这意味着学生的基本信息和行为特征对缺考行为的影响可能不大。然而, 一些特征如在读时长、形考分数和选修课程数量与缺考有轻微的正相关性, 这可能意味着这些特征较高的学生更不可能缺考。

3. 肯德尔相关系数热力图

\begin{center}
\includegraphics[max width=0.9\textwidth]{images/bo_d3vjp13ef24c73d2n190_13_250_1318_1166_751_0.jpg}
\end{center}
\hspace*{3em} 

\section*{图 3.3 肯德尔相关系数}

肯德尔相关系数是一种用于度量两个变量之间相关性的统计量,特别是当数据不满足皮尔森相关系数所需的正态分布假设时。它的值范围在-1 到 1 之间。 热力图 3.3 解释如下: age(年龄):相关系数为 0.07 ,表明年龄与缺考之间存在轻微的正相关性。年龄越大, 缺考的可能性略微降低。method (考试形式): 相关系数为 0 . ,表明考试形式与缺考之间并没有直接的线性关系。 student\_online\_day\_count(学生在线天数):相关系数为 0.23 ,表明在线天数与缺考之间存在轻微的正相关性。在线天数越多,缺考的可能性越低。 action\_count(操作次数):相关系数为 0.19 ,表明操作次数与缺考之间存在正相关性。操作次数越多,缺考的可能性降低。reply\_count(回复次数):相关系数为 0.04 , 表明回复次数与缺考之间存在轻微的正相关性。回复次数越多, 缺考的可能性略微降低。exam\_finish\_count(完成的测验数量):相关系数为 0.17, 表明完成的考试数量与缺考之间存在正相关。完成的测验数量越多, 缺考的可能性降低。finish\_rate(完成率):相关系数为 0.25 ,表明完成率越高, 缺考的可能性越低。video\_duration(视频观看时长):相关系数为 0.10,表明视频观看时长与缺考之间存在轻微的正相关性。视频观看时长越长, 缺考的可能性略微降低。select\_course\_count(选修课程数量):相关系数为-0.22,表明选修课程数量越多,缺考的可能性越高。

\section*{3.3.3 总结}

在本节, 我们使用皮尔逊、斯皮尔曼和肯德尔三种相关性分析方法进行分析是为了全面了解不同特征之间的关系, 重点关注不同特征指标对缺考与否的影响。经过分析, 上述三种相关性分析呈现了相近的结果, 即在相同的特征影响上表现出了相同的趋势, 即我们可重点查看皮尔森相关系数。

在实际应用中, 相关性分析可以帮助我们识别哪些特征可能对目标变量 (如缺考)有影响。然而,需要注意的是,相关性并不意味着因果关系。即使两个变量之间存在相关性, 也不能简单地认为一个变量的变化会导致另一个变量的变化。此外, 相关性分析只能揭示线性或单调关系, 对于非线性关系或更复杂的关系, 可能需要进一步的分析方法。在构建预测模型时, 除了考虑相关性, 还应该考虑特征的业务意义、模型的解释性以及模型的性能。

由于相关性并不等同于因果关系, 所以这需要我们对每一个特征指标与缺考的关系进行详细的数据分析。因此接下来我们将探究每个特征指标与缺考率的关系并可视化结果。
\end{document}